\documentclass[12pt]{article}

\usepackage{amsmath}
\usepackage{amssymb}
\usepackage{enumerate}
\usepackage{enumitem}
\usepackage{booktabs}
\usepackage{csquotes}
\usepackage[margin=2cm]{geometry}
\usepackage{hyperref}
\usepackage{tabularx}
\usepackage{tikz}
\usetikzlibrary{patterns, shapes.geometric, positioning, bayesnet}

\usepackage{titling}
\setlength{\droptitle}{-7em}
\usepackage{titlesec}
\titlespacing\section{0pt}{12pt plus 4pt minus 2pt}{4pt plus 2pt minus 2pt}

\title{Project Proposal \vspace{-1em}}

\author{CS396 Causal Inference}

\begin{document}

\maketitle

\section*{Instructions}

This assignment is due on Friday, April 26 at 11:59pm CDT. You cannot turn this
in late. You will have the change to turn in a revised proposal.

Please save your (group's) proposal in a single PDF named {\tt proposal.pdf}
and upload it to Canvas.
You should edit this assignment TeX to fill in your answers. This template is rather long because it includes instructions and sample answers. You should delete the instructions and just use the outline provided by the sections.

Your proposal must be \emph{at most} three pages long.

\section{Group members}

\noindent Please list your group members.

\section{Problem Statement}

\noindent Describe the problem you're considering, why it is important, and who else cares about it.

\noindent {\bf For example}, you might say: ``we want to understand how much smoking increases the risk of cardiovascular disease (CVD), which is important to public health experts because CVD is a leading cause of death.''

\section{Causal Questions or Hypotheses}

\noindent Describe at least one causal question or hypothesis you would like to investigate. 
\begin{enumerate}[itemsep=0em,label={(\alph*)}]
\item What are the treatment(s) and outcome(s)? Why?
\item Frame your question as a contrast of counterfactual random variables.
  What is the hypothetical randomized trial your question considers?
\item If you haven't been able to decide which causal question(s) you want to
  ask, please list at least two possible treatments and outcomes, plus any
  arguments for or against each.
\end{enumerate}

\noindent {\bf For example}, you might say:
\begin{enumerate}[itemsep=0em,label={(\alph*)}]
\item Our treatment A is smoking and our outcome Y is CVD.
\item We are interested in the causal risk ratio $E[Y^{a=1}] / E[Y^{a=0}]$, or
expected rate of CVD had everyone been assigned to smoke divided by the
expected rate of CVD had everyone been assigned not to smoke.
\item We might also be interested in using `Cigarettes per day' as a treatment,
which might provide a more fine-grained effect estimate, but also requires
working with a continuous-valued treatment.
\end{enumerate}

\clearpage

\section{Dataset(s)}

\noindent What dataset(s) do you plan to use?

For the following questions, answer (a) through (c) for \emph{each dataset} you
might use. You only need to answer (d) through (f) for \emph{one dataset} --
preferrably the largest dataset or the one you plan to work with first.

\begin{enumerate}[itemsep=0em,label={(\alph*)}]
\item Describe the dataset's background: how was it collected, what does it contain?
\item What are the limitations of this dataset? 
\item Describe the format of the data. Can it be represented as a NxD matrix with N individuals and D features? If not, why, and how will you handle this?
\item Load the data as a pandas dataframe (e.g. {\tt pd.read\_csv}) and provide a printout of at least three rows.
\item For at least six variables (columns) in your dataset:
  \begin{enumerate}[itemsep=0em,label={\roman*.}]
     \item Describe that variable: what is it measuring? Is it a discrete or continuous variable? Does it have any missing values? What is its mean and standard deviation?
     \item What are the possible\footnote{If you don't have the domain knowledge to answer this question, that's okay. Just focus on at least a few variables where the way the data was measured makes it clear. For example, someone's age cannot be caused by any other variables.} causal relationships between this variable and the other variables (in part e)?
  \end{enumerate}
\item What is at least one variable that your dataset doesn't contain but might be a causal factor? How might such a variable complicate your causal question(s)?
\end{enumerate}

{\bf For example}, if you were to work with the Framingham dataset, your
answers might look something like:
\begin{enumerate}[itemsep=0em,label={(\alph*)}]
\item The Framingham Heart Study was collected starting in 1948, with an
initial 5,209 subjects monitored over several years for clinical risk factors
and cardiovascular outcomes, such as $\ldots$
\item The dataset has a few important limitations. First, it has been
anonymized in such a way that makes it unsuitable for publication. Second,
$\ldots$
\item Each row of the dataset indicates an observation for a given subject, the
dataset cannot be trivially represented as an NxD matrix because not all
subjects have the same number of observations. There are a total of $X$
observations across $Y$ subjects. Each observation has $Z$ variables. 
\item After loading the dataset into pandas, we see:
\begin{verbatim}
RANDID  SEX  TOTCHOL  AGE  SYSBP  DIABP  CURSMOKE  CIGPDAY    BMI 
2448    1    195.0    39   106.0  70.0   0         0.0        26.97 
2448    1    209.0    52   121.0  66.0   0         0.0        NaN
6238    2    250.0    46   105.0  81.0   0         0.0        28.73
...
\end{verbatim}
\clearpage
\item Variables:
  \begin{enumerate}[itemsep=0em,label={\roman*.}]
     \item {\tt AGE} records the subject's age at the time of the observation.
     It is continuous, with a mean of $X$ and standard deviation of $Y$. Age
     may be a cause of most other variables in the dataset, but cannot be
     caused by anything else.
     \item {\tt SYSBP} records the subject's Systolic Blood Pressure. It is
     continuous, with a mean of $X$ and standard deviation of $Y$. We expect
     that SYSBP is caused by $\ldots$
  \end{enumerate}
\item Socioeconomic status (SES) may be a relevant but unmeasured confounder.
It likely affects all health outcomes such as $X$, and may influence our
treatment variable $Y$. Trying to account for SES will make identifying our
counterfactual $\mathbb{E}[Y^a]$ more difficult because $\ldots$
\end{enumerate}

\section{Expectations and Concerns}

Write a few sentences about what you hope to learn during this project. Are
there concepts from the class that you hope to explore with this particular
dataset? Are there any challenges you expect to encounter while working on this
project?

\section{References}

Include at least one citation for your dataset. For example:

\begin{itemize}
\item Dawber, Thomas R., Gilcin F. Meadors, and Felix E. Moore Jr. ``Epidemiological
approaches to heart disease: the Framingham Study.'' American Journal of Public
Health and the Nations Health 41.3 (1951): 279-286.
\end{itemize}

\end{document}
